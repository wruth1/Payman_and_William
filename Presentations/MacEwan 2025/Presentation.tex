\documentclass[14pt]{beamer}

\usetheme{Madrid}

% Add frame numbers
\setbeamertemplate{page number in head/foot}[framenumber]

\usepackage{amsmath, amssymb}
\usepackage{tikz}
\usepackage{bm}
\usepackage{outlines}   % For multilevel lists using the outline environment
\usepackage{natbib}
\usepackage{subcaption}

\renewcommand*{\bibfont}{\scriptsize}

\newcommand{\CMA}{Causal Mediation Analysis}

\newcommand{\bE}{\mathbb{E}}
\newcommand{\bF}{\mathbb{F}}
\newcommand{\bG}{\mathbb{G}}

\newcommand{\GLMMs}{Mixed-Effects Models}

\title[]{Selected Topics}
\author{William Ruth}
\institute[]{Université de Montréal}
\date{\vspace{-3cm}}
% \titlegraphic{\includegraphics[width=2cm]{../Logos/CANSSI_Logo.png} \hspace{2cm} \includegraphics[width=4cm]{../Logos/Logo_UdeM-CMJN.jpg}}


\begin{document}

\begin{frame}
    \titlepage
\end{frame}

\begin{frame}{Topics}
    \begin{outline}
        \1 Adaptive Pareto Smoothed Importance Sampling \newline
        \1 Multilevel Causal Mediation Analysis \newline
        \1 Modelling Tuberculosis in Foreign-Born Canadians
    \end{outline}
\end{frame}

\begin{frame}{Topics}
    \begin{outline}
        \1 \textbf{Adaptive Pareto Smoothed Importance Sampling} \newline
        \1 Multilevel Causal Mediation Analysis \newline
        \1 Modelling Tuberculosis in Foreign-Born Canadians
    \end{outline}
\end{frame}

\begin{frame}{Outline}
    \begin{outline}
        \1 Importance sampling \newline
        \1 Measuring performance \newline
        \1 Improving performance
            \2 Modifications
            \2 Optimization
    \end{outline}
\end{frame}


\begin{frame}{Importance Sampling}
    \begin{outline}
        \1 Need to compute an expected value
            \2 $\bE_F \varphi(X)$
        \1 Can't do the sum/integral \newline

        \1 Monte Carlo approximation
            \2 Simulating from $F$ might be hard
    \end{outline}
    \begin{equation*}
    \end{equation*}
\end{frame}

\begin{frame}{Importance Sampling}
    \begin{outline}
        \1 Introduce ``proposal distribution'', $G$:
    \end{outline}
    \begin{align*}
        \bE_F \varphi(X) &=  \bE_G \left[ \varphi(X) \cdot \frac{f(X)}{g(X)} \right] \\
        &=  \bE_G \left[ \varphi(X) \cdot w(X) \right]
    \end{align*}
    \begin{outline}
        \1 $G$ can be nearly anything*
            \2 *Some choices will be better than others
    \end{outline}
\end{frame}

\begin{frame}{Example: Mystery Target}
    \begin{outline}
        \1 $f$ unknown, but can be evaluated 
        \1 $\varphi(X) = X^2$ \newline

        \1 Try some proposals:
            \2 $G_1 \sim N(0, 2^2)$
            \2 $G_2 \sim N(0, 0.6^2)$ \newline
        \1 Use $M=1000$ samples from proposal
            \2 $\hat{\bE}_1 = 0.99$, $\hat{\text{SD}} = 1.97$
            \2 $\hat{\bE}_2 = 1.10$, $\hat{\text{SD}} = 2.32$
    \end{outline}    
\end{frame}

\begin{frame}{Example: Mystery Target}
    \centering
    \includegraphics[height=0.9\textheight, width=0.9\textwidth, keepaspectratio]{Figures/Wt Hist.pdf}
\end{frame}

\begin{frame}{Importance Sampling}
    \begin{outline}
        \1 $G_1$ weights look fine
        \1 $G_2$ weights dominated by one large value \newline

        \1 We can make this difference precise
        \1 ``Effective Sample Size'':
    \end{outline}
    \begin{gather*}
        ESS = \frac{\left[\sum_i w(X_i)\right]^2}{\sum_i w(X_i)^2}
    \end{gather*}
\end{frame}

\begin{frame}{Example: Mystery Target}
    \centering
    \includegraphics[height=0.7\textheight, width=0.9\textwidth, keepaspectratio]{Figures/Wt Hist.pdf} \newline
    \begin{outline}
        $ESS_1 \approx 662$ \hspace{2.5cm} $ESS_2 \approx 54$
    \end{outline}
\end{frame}

\begin{frame}{Importance Sampling}
    \begin{outline}
        \1 Problem: Low ESS $\rightarrow$ hard to estimate means \newline
        \1 But ESS is based on means
            \2 \citep{Cha18}
    \end{outline}
\end{frame}

% \begin{frame}{Importance Sampling}
%     \begin{outline}
%         \1 Consider $\varphi(X) = X$
%             \2 I.e. $\bE_F \varphi(X) = \bE_F(X)$ \newline
%         \1 $\hat{\bE} = \sum_i \frac{X_i w(X_i)}{M}$, $X_i \overset{\mathrm{iid}}{\sim} G$ \newline

%         \1 The variance of our estimator is \left( \sum_i w(X_i)^2 \right)
%     \end{outline}
% \end{frame}


\begin{frame}{Improving IS}
    \begin{outline}
        \1 Choose a good proposal \newline
        
        \1 Modify large weights        
            \2 Truncated IS
            \2 Pareto Smoothed IS
    \end{outline}
\end{frame}

\begin{frame}{Improving IS}
    \begin{outline}
        \1 Truncated Importance Sampling:
            \2 \citep{Ion08} \newline
    \end{outline}

    \setbeamertemplate{enumerate items}[default]
    \begin{enumerate}
        \item Choose a threshold
        \item Apply hard thresholding to any large weights \newline
    \end{enumerate}
    \begin{outline}
        \1 Still consistent for the target
    \end{outline}
\end{frame}

\begin{frame}{Example: Mystery Target}
    \centering
    \includegraphics[height=0.9\textheight, width=0.9\textwidth, keepaspectratio]{Figures/Wt Hist - Thresh.pdf}
\end{frame}

% \begin{frame}{Example: Mystery Target}
%     \centering
%     \includegraphics[height=0.9\textheight, width=0.9\textwidth, keepaspectratio]{Figures/Wt Hist - Trunc.pdf}
% \end{frame}

\begin{frame}{Example: Mystery Target}
    \centering
    \includegraphics[height=0.7\textheight, width=0.9\textwidth, keepaspectratio]{Figures/Wt Hist - Trunc.pdf} \newline
    \begin{outline}
        $ESS_1 \approx 662$ \hspace{2.5cm} $ESS_2 \approx 54$\\
        $ESS_1^{(\mathrm{trunc})} \approx 662$ \hspace{2.5cm} $ESS_2^{(\mathrm{trunc})} \approx 245$
    \end{outline}
\end{frame}


\begin{frame}{Improving IS}
    \begin{outline}
    \1 Pareto Smoothed Importance Sampling:
        \2 \citep{Veh24} \newline
    \end{outline}

    \setbeamertemplate{enumerate items}[default]
    \begin{enumerate}
    \item Choose a threshold
        \begin{itemize}
            \item Weights above threshold represent tail of their dist.
        \end{itemize}
    \item Approximate tail with Generalized Pareto Dist.
    \begin{itemize}
        \item Fit GPD to weights above threshold
        \item \citep{Zha09}
    \end{itemize}
    \item Replace large weights with quantiles of fitted GPD
    \end{enumerate}
\end{frame}

\begin{frame}{Example: Mystery Target}
    \centering
    \includegraphics[height=0.9\textheight, width=0.9\textwidth, keepaspectratio]{Figures/Wt Hist - Pareto Thresh.pdf}
\end{frame}

\begin{frame}{Example: Mystery Target}
    \centering
    \includegraphics[height=0.7\textheight, width=0.9\textwidth, keepaspectratio]{Figures/Wt Hist - Pareto Dens.pdf} \newline
    \begin{outline}
        $\hat{k}_1 \approx -1.81$ \hspace{2cm} $\hat{k}_2 \approx 0.72$
    \end{outline}
\end{frame}

\begin{frame}{Example: Mystery Target}
    \centering
    \includegraphics[height=0.9\textheight, width=0.9\textwidth, keepaspectratio]{Figures/Wt Hist - Pareto Dens Zoom.pdf}
\end{frame}

\begin{frame}{Example: Mystery Target}
    \centering
    \includegraphics[height=0.9\textheight, width=0.9\textwidth, keepaspectratio]{Figures/Wt Hist - Pareto Smooth Zoom.pdf}
\end{frame}

\begin{frame}{Example: Mystery Target}
    \centering
    \includegraphics[height=0.5\textheight, width=0.9\textwidth]{Figures/Wt Hist - Pareto Smooth.pdf}\newline
    \begin{outline}
        $ESS_1 \approx 662$ \hspace{2.5cm} $ESS_2 \approx 54$\\
        $ESS_1^{(\mathrm{trunc})} \approx 662$ \hspace{2.5cm} $ESS_2^{(\mathrm{trunc})} \approx 245$\\
        $ESS_1^{(\mathrm{PS})} \approx 662$ \hspace{2.5cm} $ESS_2^{(\mathrm{PS})} \approx 160$
    \end{outline}
\end{frame}


\begin{frame}{Adaptive IS}
    \begin{outline}
        \1 Alternative approach: directly optimize ESS \newline

        \1 Adaptive Importance Sampling: 
            \2 \citep{Aky21} \newline
    \end{outline}

    \begin{enumerate}
        \setbeamertemplate{enumerate items}[default]
        \item Choose a (parametric) family of proposals
        \item Iteratively update the proposal to maximize ESS
    \end{enumerate}
\end{frame}

\begin{frame}{Stochastic Approximation}
    \begin{outline}
        \1 Actually, we want to maximize a population-level analog: $\rho \approx \frac{N}{ESS}$ \newline

        \1 If we had $\rho$, we would do gradient descent
            \2 $\theta_{k+1} = \theta_k - \alpha \nabla \rho(\theta_k)$ \newline

        \1 Instead, do gradient descent on $\hat{\rho}$
            \2 $\hat{\theta}_{k+1} = \hat{\theta}_k - \alpha_k \nabla \hat{\rho}(\hat{\theta}_k)$
    \end{outline}
\end{frame}


\begin{frame}{Stochastic Approximation}
    \begin{outline}
        \1 Originally developed for root finding with noise
            \2 \citep{Rob51} \newline

        \1 Quickly adapted for optimization
            \2 Use noisy evaluations for finite difference
            \2 \citep{Kie52} 
    \end{outline}
\end{frame}

\begin{frame}{Stochastic Approximation}
    \begin{outline}
        \1 Very well developed theory
        \1 Step size $\rightarrow$ 0 \newline

        \1 Stochastic gradient descent
            \2 Resample a (very) large dataset

    \end{outline}
\end{frame}

\begin{frame}{Example: Mystery Target}
    \centering
    \includegraphics[height=0.9\textheight, width=0.9\textwidth, keepaspectratio]{Figures/ESS Traj - 0,5.pdf}
\end{frame}

\begin{frame}{Example: Mystery Target}
    \centering
    \includegraphics[height=0.9\textheight, width=0.9\textwidth, keepaspectratio]{Figures/ESS Traj - 0,9.pdf}
\end{frame}

\begin{frame}{Example: Mystery Target}
    \centering
    \includegraphics[height=0.9\textheight, width=0.9\textwidth, keepaspectratio]{Figures/ESS Traj - 2.pdf}
\end{frame}

\begin{frame}{Example: Mystery Target}
    \centering
    \includegraphics[height=0.9\textheight, width=0.9\textwidth, keepaspectratio]{Figures/ESS Traj - 10.pdf}
\end{frame}


%! This resembles the previous slide
% \begin{frame}{Example: Mystery Target}
%     \centering
%     \includegraphics[height=0.7\textheight, width=0.9\textwidth, keepaspectratio]{Figures/ESS traj.pdf} \newline
%     \begin{outline}
%         $\hat{\theta}_\mathrm{end}^{(ESS)} \approx -8 \times 10^{-4}$ \hspace{1cm} $ESS_\mathrm{end} \approx 1000 - (7 \times 10^{-4})$
%     \end{outline}
% \end{frame}


\begin{frame}{Our Method}
    \begin{outline}
        \1 Recall: Be careful using IS means to diagnose IS \newline
        
        \1 \citeauthor{Veh22} give an alternative
            \2 Shape parameter of fitted tail distribution, $\hat{k}$
    \end{outline}
\end{frame}


%! Consider adding this back in. I haven't yet found a way to present the numerics that I feel confident about
% \begin{frame}{Our Method}
%     \begin{outline}
%         \1 Consider repeated sampling with $\sigma = 0.5$ (hard)
%         \1 For each sample, estimate $\rho$ and $k$ \newline

%         \1 Coefficient of Variation:
%             \2 $\hat{\rho}$: $5.06$
%             \2 $\hat{k}$: $0.22$
%     \end{outline}
    
% %     \centering
% %   \begin{tabular}{|c|c c|}
% %     \hline
% %     & $\hat{\rho}$ & $\hat{k}$ \\
% %     \hline
% %     Mean & 29.6 & 0.65 \\ 
% %     Standard Error & 15.0 & 0.01 \\ 
% %     Coefficient of Variation & {5.1} & {0.22}\\
% %     \hline
% %   \end{tabular}
% \end{frame}

% \begin{frame}{Our Method}
%     \centering
%   \begin{tabular}{|c|c c|}
%     \hline
%     & $\hat{\rho}$ & $\hat{k}$ \\
%     \hline
%     Mean & 29.6 & 0.65 \\ 
%     Standard Error & 15.0 & 0.01 \\ 
%     \textbf{Coefficient of Variation} & \textbf{5.1} & \textbf{0.22}\\
%     \hline
%   \end{tabular}
% \end{frame}

\begin{frame}{Our Method}
    \begin{outline}
        \1 Use diagnostic as objective function \newline

        \1 Apply stochastic approximation to minimize $\hat{k}$
            \2 More precisely, its population analog: $k(\theta)$
    \end{outline}
\end{frame}


\begin{frame}{Our Method}
    \begin{outline}
        \1 Results: plot trajectories for the same values of sigma used above for ESS

        % \1 Connect back to how much worse we expect things to be in high dimensions
        %     \2 Can I do a high-dimensional version of the above problem?

        % \1 Consider cumulative averaging to emphasize differences. Don't really need to include this.
    \end{outline}
\end{frame}


% \begin{frame}{Example: Mystery Target}
%     \centering
%     \includegraphics[height=0.7\textheight, width=0.9\textwidth, keepaspectratio]{Figures/PS traj.pdf} \newline
%     \begin{outline}
%         $\hat{\theta}_\mathrm{end}^{(PS)} \approx 4 \times 10^{-3}$ \hspace{5cm}
%     \end{outline}
% \end{frame}

% \begin{frame}{Example: Mystery Target}
%     \begin{outline}
%         \1 Performance tends to be better if we average all the estimates \newline

%         \1 Call this $\bar{\theta}$
%     \end{outline}
% \end{frame}

% \begin{frame}{Example: Mystery Target}
%     \centering
%     \includegraphics[height=0.7\textheight, width=0.45\textwidth]{Figures/ESS mean traj.pdf}%
%     \includegraphics[height=0.7\textheight, width=0.45\textwidth]{Figures/PS mean traj.pdf} \newline
%     \begin{outline}
%         $\bar{\theta}_\mathrm{end}^{(ESS)} \approx -1 \times 10^{-4}$ \hspace{1.5cm} $\bar{\theta}_\mathrm{end}^{(PS)} \approx 2 \times 10^{-5}$
%     \end{outline}
% \end{frame}

% \begin{frame}{Another Example}
%     \begin{outline}
%         \1 
%     \end{outline}
% \end{frame}


\begin{frame}{Recap}
    \begin{outline}
        \1 Importance sampling and extensions
            \2 Truncation
            \2 Pareto Smoothing \newline
        
        \1 Diagnostics for importance sampling
            \2 Effective sample size
            \2 Pareto tail index \newline

        \1 Adaptive importance sampling
            \2 Stochastic approximation
    \end{outline}
\end{frame}



\begin{frame}{Topics}
    \begin{outline}
        \1 Adaptive Pareto Smoothed Importance Sampling \newline
        \1 \textbf{Multilevel Causal Mediation Analysis} \newline
        \1 Modelling Tuberculosis in Foreign-Born Canadians
    \end{outline}
\end{frame}

\begin{frame}{Topics}
    \begin{outline}
        \1 Add slides from my SSC talk
    \end{outline}
\end{frame}

\begin{frame}{Topics}
    \begin{outline}
        \1 Adaptive Pareto Smoothed Importance Sampling \newline
        \1 Multilevel Causal Mediation Analysis \newline
        \1 \textbf{Modelling Tuberculosis in Foreign-Born Canadians}
    \end{outline}
\end{frame}

\begin{frame}{Topics}
    \begin{outline}
        \1 Give brief overview and mention directions for future research
        \1 One of the profs in the department, Cristina Anton, does numerical SDEs. Mention potential collaboration
    \end{outline}
\end{frame}

\begin{frame}{Acknowledgements}
    Collaborators:
    \begin{outline}
        \1 Payman (UBC), Richard (SFU)
        \1 Bouchra (UdeM), Bruno (HEC), Rado (UdeM), Rowin (UdeM)
        \1 Jeremy (SFU), Albert (Langara) \newline
    \end{outline}

    Funding:
    \begin{outline}
        \1 CANSSI
    \end{outline}
\end{frame}



\begin{frame}
    \centering
    \Huge Thank You
\end{frame}

\begin{frame}{Some References}
    \bibliographystyle{apalike}
    \bibliography{Refs}
\end{frame}

\end{document}
