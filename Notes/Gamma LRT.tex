\documentclass{article}
\usepackage[utf8]{inputenc}
\usepackage{amsmath}
\usepackage{amssymb}
\usepackage{amsthm} 
\usepackage{xcolor}     % For the \textcolor{}{} command

\usepackage{natbib} % For citation functions like \citet and \citep


\newcommand{\bG}{\mathbb{G}}
\newcommand{\bF}{\mathbb{F}}
\newcommand{\dfdg}{d \bF / d \bG}
\newcommand{\dfdgfrac}{\frac{d \bF}{d \bG}}


\title{Gamma LRT}
\date{}


\begin{document}

\maketitle

This document describes the steps to obtain the tail function for the importance sampling weights in Gamma distribution example. 
We use an Exponential distribution as the proposal dist with pdf $g(x)$, and our target distribution is Gamma with pdf $f(x)$.

Steps are as follows (I think!)

Step 1. Set the target distribution, $f(x)$ as a Gamma distribution with $\alpha$ as shape and $\beta$ as scale:

\begin{align}
    f(x;\alpha,\beta) = \frac{\beta^{\alpha}}{\Gamma(\alpha)} x^{\alpha-1} e^{\beta x}
\end{align}

Step 2. Set the proposal distribution, $g(x)$, as an Exponential dist with scale parameter of one:
\begin{align}
    g(x) = e^{-x}
\end{align}

Step 3. The pdf of weights is $w(x;\alpha,\beta) = \frac{f(x;\alpha,\beta)}{g(x)}$

Step 4. Use the pdf in step 3 and compute its tail function as described in Pickands paper:

\begin{align}
    P(x) = \frac{F(X \geq u+x)}{F(X \geq u)}
\end{align}

where $F(.)$ is the CDF of weights as defined in step 3.

Step 5. Find $Q_3$, and median as described in the paper. Compute the $c$ and $a$. 
\end{document}