\documentclass{article}
\usepackage[utf8]{inputenc}
\usepackage{amsmath}
\usepackage{amssymb}
\usepackage{amsthm} 
\usepackage{xcolor}     % For the \textcolor{}{} command

\usepackage{natbib} % For citation functions like \citet and \citep


\newcommand{\bG}{\mathbb{G}}
\newcommand{\bF}{\mathbb{F}}
\newcommand{\dfdg}{d \bF / d \bG}
\newcommand{\dfdgfrac}{\frac{d \bF}{d \bG}}


\title{Gamma LRT}
\date{}


\begin{document}

\maketitle

This document describes the steps to obtain the tail function for the importance sampling weights in Gamma distribution example. 
We use an Exponential distribution as the proposal dist with pdf $g(x)$, and our target distribution is Gamma with pdf $f(x)$.

Steps are as follows (I think!)
\textcolor{blue}{This looks really good Payman. I just added a couple notes in blue.}

Step 1. Set the target distribution, $f(x)$ as a Gamma distribution with $\alpha$ as shape and $\beta$ as scale:

\begin{align}
    f(x;\alpha,\beta) = \frac{\beta^{\alpha}}{\Gamma(\alpha)} x^{\alpha-1} e^{\beta x}
\end{align}

Step 2. Set the proposal distribution, $g(x)$, as an Exponential dist with scale parameter of one:
\begin{align}
    g(x) = e^{-x}
\end{align}
\textcolor{blue}{More generally, we can use any scale parameter for the exponential. For the purposes of this discussion, we just need the scale parameter to be fixed. I think it's pretty reasonable to keep $\lambda=1$ for now to avoid cluttering our formulas, although eventually we'll need to do the general case.}\\

Step 3. The pdf of weights is $w(x;\alpha,\beta) = \frac{f(x;\alpha,\beta)}{g(x)}$. \textcolor{blue}{This actually isn't quite the pdf of the weights. Instead, the $X$s are iid from $g$, and the weights are obtained by evaluating $w$ at the simulated $X$s. The distribution on the weights is then induced by the distribution on $X$ and the functional relationship between $X$ and $W$. If the words ``pushforward measure'' help clarify things, that's exactly what's happening here. If not, $W$ is a transformation of $X$. In principle, we can get the pdf of $W$ by computing the Jacobian of the (inverse) transformation (i.e. the derivative of the inverse of the likelihood ratio function). Personally, I prefer to work directly with the CDF or survival function; for one, I think it's easier, and for two, we need the survival function for step 4 anyway.}\\

Step 4. Use the pdf in step 3 and compute its tail function as described in Pickands paper:

\begin{align}
    P(x) = \frac{F(X \geq u+x)}{F(X \geq u)}
\end{align}

where $F(.)$ is the CDF of weights as defined in step 3.

Step 5. Find $Q_3$, and median as described in the paper. Compute the $c$ and $a$. 
\end{document}