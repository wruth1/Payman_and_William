\documentclass{article}
\usepackage[utf8]{inputenc}
\usepackage{amsmath}
\usepackage{amssymb}
\usepackage{amsthm} 
\usepackage{xcolor}     % For the \textcolor{}{} command
\usepackage{mathrsfs}   % For the \mathscr{} font

\usepackage{natbib} % For citation functions like \citet and \citep

\usepackage{tikz}   % For graphing
\usepackage{pgfplots}   % For graphing a function

\usepackage{subcaption} % For sub-figures


\newcommand{\bG}{\mathbb{G}}
\newcommand{\bF}{\mathbb{F}}
\newcommand{\bP}{\mathbb{P}}

\newcommand{\sW}{\mathscr{W}}
\newcommand{\swe}{\mathscr{W}(\eta)}

\newcommand{\dfdg}{d \bF / d \bG}
\newcommand{\dfdgfrac}{\frac{d \bF}{d \bG}}


\title{Notes on Stochastic Approximation}
\date{}


\begin{document}

\maketitle

In order to say anything useful theoretically about our stochastic approximation algorithm, I expect to need some theoretical understanding of our tail index estimator. To this end, I have found multiple papers cataloging estimators of the Pareto tail index \citep{Gom15,Fed20}; the latter of which gives over 100 examples. Of note for later, Section 10 of \citet{Fed20} discusses a class of estimators which may be of particular interest for our purposes. Furthermore, \citet{Gom15} gives a nice framing of the theoretical considerations for extreme value theory, and Section 13 of \citet{Fed20} gives an extensive, albeit simple, Monte Carlo study.





\bibliographystyle{apalike}
\bibliography{../MyBib}

\end{document}